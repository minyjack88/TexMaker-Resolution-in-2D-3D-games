\documentclass[main.tex]{Konklusion}

\begin{document}

\chapter{Konklusion}
I dag bruger nyere PC systemer typisk et 16:9 ratio, og derfor er det fordelagtigt at bruge dette som standard, og så justere andre ratioer derefter. Forskellen på aspekt ratioer kan behjælpes på flere måder, og hvilken en der bør vælges, kommer an på hvilken type spil der bliver lavet. Hvis et spil skal være meget kompetitivt, kan letterboxes sikre at spillerne har samme syn på spil verdenen, så spillet er så retfærdigt som muligt. Hvis spillet derimod ikke er kompetitivt, eller hvis oplevelsen prioriteres højere, kan spillets FOV gøres smallere for spilleren, så hele deres skærm kan benyttes, hvilket vil se flottere ud på skærmen.

Når teksturer skal skaleres, er der mange måder at man kan lave texturen der bliver skaleret godt, fra høj opløsnings teksturer til vektorgrafik. Der er ikke en definitiv løsning, men mange spil lave høj opløsing teksture og nedskalere dem, da moderne billedprogrammer og engines er ret gode til at sørge for at nedskalere grafik ser godt ud.
  
Skærmkoordinater har mange negative følger, da de altid afhænger af skærmens opløsning, og de er derfor sjældent brugt i moderne engines og frameworks, hvilket leder til at man ikke møder dem ofte. Når man benytter dem, skal man være klar på de problemer det kan forsage, hvis man ikke implementerer det ordenligt i spillet.
\end{document}