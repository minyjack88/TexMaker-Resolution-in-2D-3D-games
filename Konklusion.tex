\documentclass[main.tex]{Konklusion}

\begin{document}

\chapter{Konklusion}
Når man konstruerer et spil, skal man tage højde for en række ting. Da alle der kommer til at spille spillet ikke bruger den samme opløsning, er det udviklerens job at sikre at alle spillerne får den bedste mulige oplevelse på deres givne system.\\*
I dag bruger nyere PC systemer typisk et 16:9 ratio, og derfor er det fordelagtigt at bruge dette som standard, og så justere andre ratioer derefter. Forskellen på aspekt ratioer kan behjælpes på flere måder, og hvilken en der bør vælges, kommer an på hvilken type spil der bliver lavet. Hvis et spil skal være meget kompetitivt, kan letterboxes sikre at spillerne har samme syn på spil verdenen, så spillet er så retfærdigt som muligt. Hvis spillet derimod ikke er kompetitivt, eller hvis oplevelsen prioriteres højere, kan spillets FOV gøres smallere for spilleren, så hele deres skærm kan benyttes, hvilket vil se flottere ud på skærmen. \\*
Når teksturer skal skaleres, er der mange måder at man kan lave texturen der bliver skaleret godt, fra høj opløsnings teksturer til vektorgrafik. Der er ikke en definitiv løsning, men mange spil lave høj opløsing teksture og nedskalere dem, da moderne billedprogrammer og engines er ret gode til at sørge for at nedskalere grafik ser godt ud.\\*  
Skræm koordinater har mange negativ følger da de altid afhænger af skærmens  og  derfor ikke brugt i meget modern engine og frameworks, hvilket ledere til at man ikke mødere dem ofte men når man møder dem skal være klar på de problemer de kan forårsage hvis man ikke indtager det ordentligt i sit spil. 
\end{document}