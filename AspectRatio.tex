\documentclass[Main.tex]{AspectRatio.tex}
\begin{document}
\chapter{Aspekt ratio}

Aspekt ratio refererer til den proportionelle forskel mellem en skærms højde og brede \cite{Gibson}. Der findes mange forskellige aspekt ratioer, som bliver brugt til forskellige formål \cite{CommonResolutions}. I dag er de langt mest benyttede opløsning blandt spillere på Steam 1920x1080, efterfulgt af 1366x768. Begge disse opløsninger er 16:9 formater. Andre formater der udgør en betydelig del er 1280x1024 (5:4), 1680x1050 (16:10), og 1440x900 (16:10) \cite{Steam}.

At skifte mellem to forskellige opløsninger, som deler det samme aspekt ratio, er relativt simpelt, da teksturer blot kan skaleres til at have en anden størrelse. Så længe at ratioen er den samme, og teksturen bliver skaleret ned til en dårligere opløsning, er der generelt ikke et problem. 

Problematikken opstår når at aspekt ratioen ændres, da en skalering ikke længere blot vil ændre på størrelsen af teksturen, men også på forholdet mellem bredden og højde. Det betyder at akserne ikke bliver skaleret synkront. Dette har været tilfældet i mange ældre spil, der originalt ikke var lavet til widescreen, blandt andet Warcraft 3\cite{Warcraft3Game}, som kan ses på figur \ref{Warcraft_4,3} og \ref{Warcraft_16,9}. \cite{Wills} Der er tre måder at komme uden om dette problem på, som bliver omtalt i dette afsnit.

\begin{figure}[h]
\centering
\parbox{7cm}{   
\includegraphics[width = 7cm]{billeder/Warcraft_4,3}
\caption{Warcraft 3 i dets originale 4:3 aspekt ratio}    
\label{Warcraft_4,3}}
\qquad
\begin{minipage}{9.33cm}
\includegraphics[width = 9.33cm]{billeder/Warcraft_16,9}
\caption{Warcraft 3 skaleret asynkront til et 16:9 aspekt ratio}    
\label{Warcraft_16,9}
\end{minipage}
\end{figure}

\section{Field of View}

Den ene måde at komme uden om problemet på, er at ændre på field of view (FOV), som er hvor stor en del af spilverdenen spilleren ser. Dette bringer imidlertidigt sine egne problemer på banen. Når FOV ændres, ændres der også på det output spilleren får på sin skærm, hvilket betyder at alle spillere ikke ser spilverdenen ens. Hvis spillet er kompetitivt, medfører det med stor sandsynlighed at spillere med et vist aspekt ratio har en fordel over dem med et andet aspekt ratio \cite{Atwood}.

Hvis vi ser på et 2d spil, og forestiller os et endless runner spil, så som Robot Unicorn Attack\cite{RUAGame}, hvor at man kun ser hvad der er lige foran sin karakter, vil et bredere FOV betyde at man har længere tid til at forberede sig på de forhindringer der kommer forude.

Hvis vi ser på et 3d spil, så som CS:GO \cite{CS:GOGame}, vil et bredere field of view betyde at spilleren kan se længere til siderne. Hvilket giver en stor fordel når det kommer til at få visuel kontakt med sine fjender. På figur \ref{CS_GO_16,9} ses f.eks. en dør til højre, som ikke kan ses på figur \ref{CS_GO_4,3} pga. forskellen i FOV.

Når hele spillet foregår på en flade hvorpå spillerens synsvinkel ikke ændre sig, kan man tage udgangspunkt i midten af sin baggrund, og dertil sørge for at lave en baggrund bred nok til at supportere et bredt aspekt ratio, og høj nok til at supportere et smalt et. På den måde sikres det passe samt give et godt indtryk på ethvert aspekt ratio. 

Det online kortspil Hearthstone \cite{HSGame} har grafik der går ud over selve den kvadratiske brugerflade, således at grafikken på en widescreen skærm også giver et godt visuelt indtryk, og ikke har et behov for at benytte sig af pillarboxes. På figur \ref{Hearthstone_4,3} og \ref{Hearthstone_16,9} ses det hvordan Hearthstone tilpasser sig den givne skærms aspekt ratio.

\section{Letterbox}

Den anden måde at komme uden om skaleringsproblemer på, er at tilføje sorte bjælker, også kaldet letterbox, til toppen og bunden af skærmen. På den måde tvinges spillet reelt set til at køre i et statisk aspekt ratio, og dermed er der ingen uretfærdigheder når det kommer til FOV \cite{computerhope}. Disse sorte bjælker kan også blive smidt på højre og venstre side af skærmen, og i disse tilfælde kaldes det pillarboxing \cite{Apple}. Nogle spil er nødt til at benytte sig af dette koncept, for at deres gameplay kan hænge sammen.

Et af disse spil er Speedrunners \cite{SpeedrunnersGame}. Alle fire spillere ser på den samme del af spilverdenen, og når at en spiller falder for langt bagefter, og kommer uden for skærmen, bliver spilleren elimineret. I det at spillet afhænger af at en spiller kommer uden for skærmen, ville forskellige aspekt ratioer betyde at dem med et smallere FOV enten ville dø tidligere, eller vil være ude af stand til at se sin karakter mens han stadig er i live. Ved at tilføje letterboxes til spillere der ikke kører med et 16:9 aspekt ratio, sikre spillet sig at alle har det samme syn på spil verdenen, og der er enighed om hvornår at en spiller er ude af skærmen. På figur \ref{Speedrunners_16,9} ses spillet i det tiltænkte 16:9 aspekt ratio, og på figur \ref{Speedrunners_4,3} ses spillet i et 4:3 ratio hvor det er letterboxed.

\begin{figure}[h]
\centering
\parbox{7cm}{
\includegraphics[width = 7cm]{billeder/Heathstone_4,3}
\caption{Hearthstone med aspekt ratio 4:3}
\label{Hearthstone_4,3}}
\qquad
\begin{minipage}{9.33cm}
\includegraphics[width = 9.33cm]{billeder/Heathstone_16,9}
\caption{Hearthstone med aspekt ratio 16:9}
\label{Hearthstone_16,9}
\end{minipage}
\end{figure}
\end{document}

\begin{figure}[h]
\centering
\parbox{7cm}{
\includegraphics[width = 7cm]{billeder/CS_GO_4,3}
\caption{CS:GO med aspekt ratio 4:3}    
\label{CS_GO_4,3}}
\qquad
\begin{minipage}{9.33cm}
\includegraphics[width = 9.33cm]{billeder/CS_GO_16,9}
\caption{CS:GO med aspekt ratio 16:9}    
\label{CS_GO_16,9}
\end{minipage}
\end{figure} 

\begin{figure}[h]
\centering
\parbox{7cm}{
\includegraphics[width = 7cm]{billeder/Speedrunners_4,3}
\caption{Speedrunners med aspekt ratio 4:3}    
\label{Speedrunners_4,3}}
\qquad
\begin{minipage}{9.33cm}
\includegraphics[width = 9.33cm]{billeder/Speedrunners_16,9}
\caption{Speedrunners med aspekt ratio 16:9}    
\label{Speedrunners_16,9}
\end{minipage}
\end{figure}
