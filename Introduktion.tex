\documentclass[Main.tex]{introduktion}

\begin{document}

\chapter{Introduktion}

				
\section{Abstract}

\section{Problemformulering}
Hvilke tekniske komplikationer er forbundet med skærmopløsning i forhold til 2d/3d spil og hvordan kan de forebygges?
Hvilke komplikationer kan der opstå når et spil skal kunne bruges på skærme med forskellige aspekt ratioer?
Hvilke komplikationer opstår når teksturer skal tilpasses forskellige skærmopløsninger?
Hvilke tekniske udfordringer kan der opstå når man bruger skærmen som koordinatsystem?
		
\section{Indledning}

Når man skal konstruerer et spil, skal det typisk kunne spilles på mange forskellige platformer med forskelligt hardware, og for at alle spillerne på de forskellige plaformer, skal kunne få den bedst mulig oplevelse til det hardware de har, bliver man nødt til at have forskellige skærmopløsninger. 
Når man har forskellige skærmopløsning i sit spil, kan der opstå en række problemer, som man som udvikler bliver nødt til at løse. Disse problemer kan summeres ned til tre: Skalering, forskellige aspekt ratio, og pixelkoordinater.		
	
		
\section{Forord}
Denne artikel er lavet til spil datamatikernes akademisk formidlings projekt / E-bogsantologi for EA Dania. Artiklen er skrevet i tidsrummet fra mandag d. 16 marts til onsdag d. 1 april af tre spil datamatiker studerende ved EA Dania i Grenaa.
Emnet omhandler hvilke komplikationer der kan opstå i forbindelse med valg af skærmopløsning i 2d og 3d spil, såvel som hvordan det kan forebygges. Der vil i denne artikel blive diskuteret de centrale problemstillinger der opstår, når at et spil skal kunne tilpasse sig mange forskellige skærmopløsninger, og hvordan det kan påvirke spilleren. 
Det antages at læseren har en basal forståelse for spilprogrammering og koncepter involveret i spildesign.

\end{document}